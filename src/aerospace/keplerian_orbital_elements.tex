% References:
%   [1] Pier Paolo's answer on https://tex.stackexchange.com/questions/386030/how-to-draw-orbital-elements
%   [2] https://tex.stackexchange.com/questions/250719/drawing-a-shaded-hemisphere-using-tikz

% ---------
% Preamble.
% ---------

% Document type.
\documentclass{article}

% Import custom style.
\usepackage{../.preamble/tikz_diagrams_template}

% Color theme (black, red, blue, green, orange, purple, gold).
\colortheme{blue}

% ---------
% Document.
% ---------

\begin{document}

    % Sets viewing orientation (declination/rotation) of 3D coordinate system.
    \tdplotsetmaincoords{70}{110}

    % -----------------
    % TikZ environment.
    % -----------------

    \begin{tikzpicture}[tdplot_main_coords,scale=4]
        
        % -----------------
        % Orbital elements.
        % -----------------
        
        \pgfmathsetmacro{\semimajoraxis}{1}     % semi-major axis [DU]
        \pgfmathsetmacro{\eccentricity}{0.3}    % eccentricity [-]
        \pgfmathsetmacro{\inclination}{40}      % inclination [deg]
        \pgfmathsetmacro{\raan}{25}             % right ascension of ascending node [deg]
        \pgfmathsetmacro{\argper}{80}           % argument of periapsis [deg]
        \pgfmathsetmacro{\trueanomaly}{60}      % true anomaly [deg]

        % -------------
        % Calculations.
        % -------------
        
        % eccentric anomaly [deg]
        \pgfmathsetmacro{\eccentricanomaly}{2*atan(sqrt((1-\eccentricity)/(1+\eccentricity))*tan(\trueanomaly/2))}
        
        % position in perifocal frame [DU]
        \pgfmathsetmacro{\rP}{\semimajoraxis*(cos(\eccentricanomaly)-\eccentricity)}
        \pgfmathsetmacro{\rQ}{\semimajoraxis*sqrt(1-\eccentricity^2)*sin(\eccentricanomaly)}
        \pgfmathsetmacro{\rW}{0}
        
        % velocity in perifocal frame (mean motion = 1) [DU/TU]
        \pgfmathsetmacro{\coefficient}{(\semimajoraxis)/(1-\eccentricity*cos(\eccentricanomaly))}
        \pgfmathsetmacro{\vP}{\coefficient*(-sin(\eccentricanomaly))}
        \pgfmathsetmacro{\vQ}{\coefficient*sqrt(1-\eccentricity^2)*cos(\eccentricanomaly)}
        \pgfmathsetmacro{\vW}{0}
        
        % position in ECI frame [DU]
        \pgfmathsetmacro{\rI}{\rQ*(-cos(\raan)*sin(\argper)-cos(\inclination)*cos(\argper)*sin(\raan))+\rP*(cos(\argper)*cos(\raan)-cos(\inclination)*sin(\argper)*sin(\raan))}
        \pgfmathsetmacro{\rJ}{\rP*(cos(\inclination)*cos(\raan)*sin(\argper)+cos(\argper)*sin(\raan))+\rQ*(cos(\inclination)*cos(\argper)*cos(\raan)-sin(\argper)*sin(\raan))}
        \pgfmathsetmacro{\rK}{\rQ*cos(\argper)*sin(\inclination)+\rP*sin(\inclination)*sin(\argper)}
        
        % velocity in ECI frame [DU/TU]
        \pgfmathsetmacro{\vI}{\vQ*(-cos(\raan)*sin(\argper)-cos(\inclination)*cos(\argper)*sin(\raan))+\vP*(cos(\argper)*cos(\raan)-cos(\inclination)*sin(\argper)*sin(\raan))}
        \pgfmathsetmacro{\vJ}{\vP*(cos(\inclination)*cos(\raan)*sin(\argper)+cos(\argper)*sin(\raan))+\vQ*(cos(\inclination)*cos(\argper)*cos(\raan)-sin(\argper)*sin(\raan))}
        \pgfmathsetmacro{\vK}{\vQ*cos(\argper)*sin(\inclination)+\vP*sin(\inclination)*sin(\argper)}
        
        % angular momentum vector in ECI frame [DU^2/TU]
        \pgfmathsetmacro{\hIvec}{\rJ*\vK-\rK*\vJ}
        \pgfmathsetmacro{\hJvec}{\rK*\vI-\rI*\vK}
        \pgfmathsetmacro{\hKvec}{\rI*\vJ-\rJ*\vI}
        
        % normalizing ECI angular momentum vector for plotting
        \pgfmathsetmacro{\hI}{\hIvec/sqrt((\hIvec)^2+(\hJvec)^2+(\hKvec)^2)}
        \pgfmathsetmacro{\hJ}{\hJvec/sqrt((\hIvec)^2+(\hJvec)^2+(\hKvec)^2)}
        \pgfmathsetmacro{\hK}{\hKvec/sqrt((\hIvec)^2+(\hJvec)^2+(\hKvec)^2)}
        
        % node vector
        \pgfmathsetmacro{\nI}{-\hJ}
        \pgfmathsetmacro{\nJ}{\hI}
        \pgfmathsetmacro{\nK}{0}
        
        % semi-latus rectum [DU]
        \pgfmathsetmacro{\semilatusrectum}{\semimajoraxis*(1-\eccentricity^2)}
        
        % eccentricity vector [-]
        \pgfmathsetmacro{\rdotv}{\rI*\vI+\rJ*\vJ+\rK*\vK}
        \pgfmathsetmacro{\rmag}{sqrt((\rI)^2+(\rJ)^2+(\rK)^2)}
        \pgfmathsetmacro{\vmag}{sqrt((\vI)^2+(\vJ)^2+(\vK)^2)}
        \pgfmathsetmacro{\eI}{((\vmag)^2-(1/\rmag))*\rI-\rdotv*\vI}
        \pgfmathsetmacro{\eJ}{((\vmag)^2-(1/\rmag))*\rJ-\rdotv*\vJ}
        \pgfmathsetmacro{\eK}{((\vmag)^2-(1/\rmag))*\rK-\rdotv*\vK}
        
        % -------------------
        % Drawing the planes.
        % -------------------
        
        % portion of orbital plane below the equatorial plane (shaded)
        \draw[fill=color_theme,opacity=0.25]plot[variable=\x,domain=(180-\argper):(360-\argper),samples=360]({sqrt(((\semilatusrectum*cos(\x)/(1+\eccentricity*cos(\x)))^2)+((\semilatusrectum*sin(\x)/(1+\eccentricity*cos(\x)))^2))*(cos(\raan)*cos(\argper+\x)-sin(\raan)*sin(\argper+\x)*cos(\inclination))},{sqrt(((\semilatusrectum*cos(\x)/(1+\eccentricity*cos(\x)))^2)+((\semilatusrectum*sin(\x)/(1+\eccentricity*cos(\x)))^2))*(sin(\raan)*cos(\argper+\x)+cos(\raan)*sin(\argper+\x)*cos(\inclination))},{sqrt(((\semilatusrectum*cos(\x)/(1+\eccentricity*cos(\x)))^2)+((\semilatusrectum*sin(\x)/(1+\eccentricity*cos(\x)))^2))*sin(\argper+\x)*sin(\inclination)});
        
        % portion of orbital plane below the equatorial plane (outline)
        \draw[color_theme,thick]plot[variable=\x,domain=(180-\argper):(360-\argper),samples=360]({sqrt(((\semilatusrectum*cos(\x)/(1+\eccentricity*cos(\x)))^2)+((\semilatusrectum*sin(\x)/(1+\eccentricity*cos(\x)))^2))*(cos(\raan)*cos(\argper+\x)-sin(\raan)*sin(\argper+\x)*cos(\inclination))},{sqrt(((\semilatusrectum*cos(\x)/(1+\eccentricity*cos(\x)))^2)+((\semilatusrectum*sin(\x)/(1+\eccentricity*cos(\x)))^2))*(sin(\raan)*cos(\argper+\x)+cos(\raan)*sin(\argper+\x)*cos(\inclination))},{sqrt(((\semilatusrectum*cos(\x)/(1+\eccentricity*cos(\x)))^2)+((\semilatusrectum*sin(\x)/(1+\eccentricity*cos(\x)))^2))*sin(\argper+\x)*sin(\inclination)});
        
        % equatorial plane
        \filldraw[draw=black,fill=gray!20](-1.1,-1,0)--(-1.1,1,0)--(1,1,0)--(1,-1,0)--cycle;
        \node at (0.8*\semimajoraxis,-0.25*\semimajoraxis,0)[left,rotate=-7]{equatorial plane};
        
        % portion of orbital plane above the equatorial plane (shaded)
        \draw[fill=color_theme,opacity=0.25]plot[variable=\x,domain=-\argper:(180-\argper),samples=360]({sqrt(((\semilatusrectum*cos(\x)/(1+\eccentricity*cos(\x)))^2)+((\semilatusrectum*sin(\x)/(1+\eccentricity*cos(\x)))^2))*(cos(\raan)*cos(\argper+\x)-sin(\raan)*sin(\argper+\x)*cos(\inclination))},{sqrt(((\semilatusrectum*cos(\x)/(1+\eccentricity*cos(\x)))^2)+((\semilatusrectum*sin(\x)/(1+\eccentricity*cos(\x)))^2))*(sin(\raan)*cos(\argper+\x)+cos(\raan)*sin(\argper+\x)*cos(\inclination))},{sqrt(((\semilatusrectum*cos(\x)/(1+\eccentricity*cos(\x)))^2)+((\semilatusrectum*sin(\x)/(1+\eccentricity*cos(\x)))^2))*sin(\argper+\x)*sin(\inclination)});
        
        % portion of orbital plane above the equatorial plane (outline)
        \draw[color_theme,thick]plot[variable=\x,domain=-\argper:(180-\argper),samples=360]({sqrt(((\semilatusrectum*cos(\x)/(1+\eccentricity*cos(\x)))^2)+((\semilatusrectum*sin(\x)/(1+\eccentricity*cos(\x)))^2))*(cos(\raan)*cos(\argper+\x)-sin(\raan)*sin(\argper+\x)*cos(\inclination))},{sqrt(((\semilatusrectum*cos(\x)/(1+\eccentricity*cos(\x)))^2)+((\semilatusrectum*sin(\x)/(1+\eccentricity*cos(\x)))^2))*(sin(\raan)*cos(\argper+\x)+cos(\raan)*sin(\argper+\x)*cos(\inclination))},{sqrt(((\semilatusrectum*cos(\x)/(1+\eccentricity*cos(\x)))^2)+((\semilatusrectum*sin(\x)/(1+\eccentricity*cos(\x)))^2))*sin(\argper+\x)*sin(\inclination)});
        
        % projection of "upper" orbital plane onto equatorial plane (dotted line)
        \draw[dotted]plot[variable=\x,domain=-\argper:(180-\argper),samples=360]({sqrt(((\semilatusrectum*cos(\x)/(1+\eccentricity*cos(\x)))^2)+((\semilatusrectum*sin(\x)/(1+\eccentricity*cos(\x)))^2))*(cos(\raan)*cos(\argper+\x)-sin(\raan)*sin(\argper+\x)*cos(\inclination))},{sqrt(((\semilatusrectum*cos(\x)/(1+\eccentricity*cos(\x)))^2)+((\semilatusrectum*sin(\x)/(1+\eccentricity*cos(\x)))^2))*(sin(\raan)*cos(\argper+\x)+cos(\raan)*sin(\argper+\x)*cos(\inclination))},0);
        
        % orbital plane label
        \node at (0.7*\semimajoraxis,-0.52*\semimajoraxis,-0.25*\semimajoraxis)[left,text width=4em,color_theme]{orbital plane};

        % --------------------
        % Drawing the vectors.
        % --------------------
        
        % inertial (IJK) axes
        \draw[-mylatex',/tiplen=0.3cm,thick](0,0,0)--(1.5,0,0)node[pos=0.9,xshift=10]{$\boldsymbol{\hat{I}}$};
        \draw[-mylatex',/tiplen=0.3cm,thick](0,0,0)--(0,1.5,0)node[pos=1.03]{$\boldsymbol{\hat{J}}$};
        \draw[-mylatex',/tiplen=0.3cm,thick](0,0,0)--(0,0,1.5)node[pos=1.05]{$\boldsymbol{\hat{K}}$};
        
        % vernal equinox direction
        \draw[dashed](1.5,0,0)--(3,0,0)node[pos=1.55,rotate=40,yshift=2]{vernal equinox ($\vernal$)};
        
        % angular momentum vector
        \draw[-mylatex',/tiplen=0.3cm,thick](0,0,0)--(\hI,\hJ,\hK)node[above]{$\boldsymbol{\hat{h}}$};
        
        % Position vector.
        \draw[-mylatex',/tiplen=0.3cm,thick](0,0,0)--(\rI,\rJ,\rK)node[pos=0.8,right]{$\boldsymbol{r}$};
        
        % node vector (scaled)
        \draw[-mylatex',/tiplen=0.3cm,thick](0,0,0)--(2.25*\nI,2.25*\nJ,2.25*\nK)node[pos=1,xshift=-10,yshift=5]{$\boldsymbol{n}$};
        
        % line of nodes
        \draw[dotted,thick](-3*\nI,-3*\nJ,-3*\nK)--(3*\nI,3*\nJ,3*\nK)node[pos=1.2,rotate=-76]{line of nodes};
        
        % ascending node
        \draw[thick,black,fill=black](1.36*\nI,1.36*\nJ,1.36*\nK)circle(0.1ex)node[xshift=-8]{$\ascnode$};
        
        % descending node
        \draw[thick,black,fill=black](-1.48*\nI,-1.48*\nJ,-1.48*\nK)circle(0.1ex)node[xshift=-8]{$\descnode$};
        
        % eccentricity vector (scaled)
        \draw[-mylatex',/tiplen=0.3cm,thick](0,0,0)--(2*\eI,2*\eJ,2*\eK)node[pos=1,right,xshift=-5,yshift=-4]{$\boldsymbol{e}$};
        
        % satellite position
        \draw[thick,black,fill=white](\rI,\rJ,\rK)circle(0.15ex)node[align=left,xshift=8,yshift=14]{satellite\\position};
        
        % line of apsides
        \draw[dotted,thick](0,0,0)--(3*\eI,3*\eJ,3*\eK)node[pos=1.45,rotate=35,align=left,yshift=-8]{line of apsides \\(periapsis direction)};
        \draw[dotted,thick](-5*\eI,-5*\eJ,-5*\eK)--(-1.9*\eI,-1.9*\eJ,-1.9*\eK);
        
        % -------------------
        % Drawing the angles.
        % -------------------
        
        % right ascension of the ascending node
        \tdplotdrawarc[-mylatex',/tiplen=0.2cm]{(0,0,0)}{0.33*\semimajoraxis}{0}{\raan}{anchor=north}{$\Omega$}
        
        % rotates TikZ's coordinate system for drawing argument of periapsis
        \tdplotsetrotatedcoords{\raan-90}{\inclination}{0}

        % argument of periapsis and true anomaly
        \begin{scope}[tdplot_rotated_coords]

            % argument of periapsis
            \tdplotdrawarc[-mylatex',/tiplen=0.2cm]{(0,0,0)}{0.33*\semimajoraxis}{90}{90+\argper}{anchor=west}{$\omega$}
            
            % true anomaly
            \tdplotdrawarc[-mylatex',/tiplen=0.2cm]{(0,0,0)}{0.33*\semimajoraxis}{90+\argper}{90+\argper+\trueanomaly}{anchor=south west}{$\nu$}
            
        \end{scope}

        % inclination (K to h)
        %   --> 1st coordinate: center of orbit
        %   --> 2nd coordinate: start of arc
        %   --> 3rd coordinate: end of arc
        %   --> use radius of sphere for radius of arc
        \tdplotdefinepoints(0,0,0)(0,0,1)(\hI,\hJ,\hK)
        \tdplotdrawpolytopearc[-mylatex',/tiplen=0.2cm]{0.8*\semimajoraxis}{anchor=south}{$i$}
        
        % inclination (between planes)
        %   --> 1st coordinate: center of orbit
        %   --> 2nd coordinate: start of arc
        %   --> 3rd coordinate: end of arc
        %   --> use radius of sphere for radius of arc
        \tdplotdefinepoints(0,0,0)(0.5,0.5,0)(0.57,0.48,0.14)
        \tdplotdrawpolytopearc[-mylatex',/tiplen=0.2cm]{0.75*\semimajoraxis}{anchor=west,yshift=2}{$i$}
        
    \end{tikzpicture}

\end{document}