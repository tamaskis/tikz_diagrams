% References:
%   [1] https://tex.stackexchange.com/questions/118069/how-to-draw-an-euler-angle-rotation-sequence-with-tikz
%   [2] https://tex.stackexchange.com/questions/218354/i-want-to-draw-a-dot-small-sphere-in-a-tikz-3d-plot

% ---------
% Preamble.
% ---------

% document type
\documentclass{article}

% import custom style
\usepackage{../.preamble/tikz_diagrams_template}

% color theme (black, red, blue, green, orange, purple, gold)
\colortheme{blue}

% ---------
% Document.
% ---------

\begin{document}
    
    % sets viewing orientation (declination/rotation) of 3D coordinate system
    \tdplotsetmaincoords{60}{140}
    
    % change rotation matrix to use 3-2-1 sequence
    \tdseteulerxyz
    
    % -----------------
    % TikZ environment.
    % -----------------
    
    \begin{tikzpicture}[tdplot_main_coords,scale=1]
    
        % axes length
        \pgfmathsetmacro{\axlen}{3}
        
        % frame A axes
        \draw[](0,0,0)--(\axlen,0,0)node[pos=1.1]{$x$};
        \draw[](0,0,0)--(0,\axlen,0)node[pos=1.12]{$y$};
        \draw[](0,0,0)--(0,0,\axlen)node[pos=1.08]{$z$};
    
        % frame A origin
        \node[xshift=-1,yshift=-7]at(0,0){$O$};
    
        % frame A label
        \node[]at(0,-1,0){$\mathcal{A}$};
    
        % rotation angles [deg]
        \pgfmathsetmacro{\zRot}{170}
        \pgfmathsetmacro{\yRot}{-10}
        \pgfmathsetmacro{\xRot}{180}
        
        % frame B origin
        \pgfmathsetmacro{\Ox}{5}
        \pgfmathsetmacro{\Oy}{-3}
        \pgfmathsetmacro{\Oz}{0}
            
        % rotates TikZ's coordinate system to align with body frame
        \tdplotsetrotatedcoords{\zRot}{\yRot}{\xRot}
        
        % frame B axes
        \draw[tdplot_rotated_coords](\Ox,\Oy,\Oz)--(\Ox+\axlen,\Oy,\Oz)node[pos=1.08]{$a$};
        \draw[tdplot_rotated_coords](\Ox,\Oy,\Oz)--(\Ox,\Oy+\axlen,\Oz)node[pos=1.12]{$b$};
        \draw[tdplot_rotated_coords](\Ox,\Oy,\Oz)--(\Ox,\Oy,\Oz+\axlen)node[pos=1.08]{$c$};
    
        % frame B origin
        \node[xshift=2,yshift=7,tdplot_rotated_coords]at(\Ox,\Oy,\Oz){$O'$};
    
        % frame B label
        \node[tdplot_rotated_coords]at(\Ox+1,\Oy+1,\Oz){$\mathcal{B}$};
    
        % frame B origin in frame A coordinates
        \pgfmathsetmacro{\OxA}{-5.370176084965562}
        \pgfmathsetmacro{\OyA}{-2.099372900722452}
        \pgfmathsetmacro{\OzA}{0.868240888334651}
        
        % position of frame B w.r.t. frame A
        \draw[-mylatex',thick](0,0,0)--(\OxA,\OyA,\OzA)node[pos=0.5,xshift=17]{$\boldsymbol{r}_{\mathcal{B}/\mathcal{A}}$};
    
        % particle coordinates
        \pgfmathsetmacro{\Px}{-4}
        \pgfmathsetmacro{\Py}{-3}
        \pgfmathsetmacro{\Pz}{2.5}
    
        % particle
        \node[draw=none,shape=circle,fill,inner sep=1.75pt](d1)at(\Px,\Py,\Pz){};
        \node[yshift=9]at(\Px,\Py,\Pz){$P$};
        
        % position of particle w.r.t. frame A
        \draw[-mylatex',thick](0,0,0)--(\Px,\Py,\Pz)node[pos=0.8,xshift=-13]{$\boldsymbol{r}_{P/\mathcal{A}}$};
    
        % position of particle w.r.t. frame B
        \draw[-mylatex',thick,color_theme](\OxA,\OyA,\OzA)--(\Px,\Py,\Pz)node[pos=0.5,xshift=10,yshift=5]{$\boldsymbol{r}_{P/\mathcal{B}}$};
        
    \end{tikzpicture}

\end{document}