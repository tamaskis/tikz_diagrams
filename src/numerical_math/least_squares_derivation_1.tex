% References:
%   [1] https://tex.stackexchange.com/questions/126019/drawing-a-plane-in-3d-space

% ---------
% Preamble.
% ---------

% Document type.
\documentclass{article}

% Import custom style.
\usepackage{../.preamble/tikz_diagrams_template}

% Color theme (black, red, blue, green, orange, purple, gold).
\colortheme{blue}

% ---------
% Document.
% ---------

\begin{document}

    % Sets viewing orientation (declination/rotation) of 3D coordinate system.
    \tdplotsetmaincoords{70}{110}

    % -----------------
    % TikZ environment.
    % -----------------

    \begin{tikzpicture}[scale=3,tdplot_main_coords]
        
        % plane
        \filldraw[draw=color_theme,fill=shade_color](0,0,0)--(0,2,0)--(2,2,0)--(2,0,0)--cycle
            node[right=185,pos=0.4,color_theme]{$C(\mathbf{A})$}
        ;
        
        % vector (b)
        \draw[-mylatex',thick,/tiplen=0.3cm](1.5,0.5,0)--(0.5,1.5,.75)node[above=12,pos=0.55]{$\mathbf{b}\notin C(\mathbf{A})$};
        
        % vector parallel to C(A) (Ax)
        \draw[-mylatex',thick,/tiplen=0.3cm](1.5,0.5,0)--(0.5,1.5,0)node[below=3,pos=0.6]{$\mathbf{A}\mathbf{x}\in C(\mathbf{A})$};
        
        % vector normal to C(A) (b-Ax)
        \draw[-mylatex',thick,/tiplen=0.3cm](0.5,1.5,0)--(0.5,1.5,.75)node[right,midway]{$\mathbf{b}-\mathbf{A}\mathbf{x}\in C(\mathbf{A})^{\perp}$};
        
        % perpendicular symbol
        \draw[thin](0.5,1.5,0)--(0.5,1.5,0.1)--(0.55,1.45,0.1)--(0.55,1.45,0)--cycle;
        
    \end{tikzpicture}

\end{document}