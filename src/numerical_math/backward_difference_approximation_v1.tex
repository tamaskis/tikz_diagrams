% ---------
% Preamble.
% ---------

% document type
\documentclass{article}

% import custom style
\usepackage{../.preamble/tikz_diagrams_template}

% color theme (black, red, blue, green, orange, purple, gold)
\colortheme{blue}

% ---------
% Document.
% ---------

\begin{document}

    % ------------------------------------------
    % TikZ environment and function definitions.
    % ------------------------------------------

    \begin{tikzpicture}[declare function={
        f(\x)=\x/5-cos(deg(\x*1.85))/2+1.5;
        xa=4;
        xb=5;
        g(\x)=f(xa)+((f(xb)-f(xa))/(xb-xa))*(\x-xa);
    }]
        
        % plots function from x = 1 to x = 9
        \draw[smooth,samples=100,domain=1:9,thick,color_theme]plot(\x,{f(\x)})node[anchor=south,xshift=-10,yshift=3]{$y=f(x)$};
        
        % draws axes
        \draw(0,0)--(10,0)node[pos=1,xshift=5]{$x$};
        \draw(0,0)--(0,5)node[pos=1,yshift=5]{$y$};
        
        % plots secant line for forward difference approximation
        \draw[smooth,samples=100,domain=(xa-0.5):(xb+0.5),dashed,thick]plot(\x,{g(\x)});
        
        % evaluates function at evaluation points
        \pgfmathsetmacro{\fa}{f(xa)};
        \pgfmathsetmacro{\fb}{f(xb)};
        
        % dotted lines to evaluation points
        \draw[dotted](xa,0)--(xa,\fa);
        \draw[dotted](xb,0)--(xb,\fb);
        
        % tick marks on x-axis
        \draw(xa,0)--(xa,-0.15);
        \draw(xb,0)--(xb,-0.15);
        
        % x-axis labels
        \node at(xa,-0.35){$x_{0}-h$};
        \node at(xb,-0.4){$x_{0}$};
        
        % plots evaluation points
        \draw[black,fill=white,thick](xa,\fa)circle(0.08)node[black,above,xshift=-10,yshift=18]{$f(x_{0}-h)$};
        \draw[black,fill=white,thick](xb,\fb)circle(0.08)node[black,above,yshift=18]{$f(x_{0})$};
        
        % lines from evaluation points to labels
        \draw[gray](xa,\fa+0.12)--(xa,\fa+0.7);
        \draw[gray](xb,\fb+0.12)--(xb,\fb+0.7);
        
    \end{tikzpicture}

\end{document}